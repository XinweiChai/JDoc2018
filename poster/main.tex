%%%%%%%%%%%%%%%%%%%%%%%%%%%%%%%%%%%%%%%%%%%%%%%%%%%%%%%%%%%%%%%%%%%%%%%%%%%%%%
%% Beamer style for JDOC A1 poster
%%%%%%%%%%%%%%%%%%%%%%%%%%%%%%%%%%%%%%%%%%%%%%%%%%%%%%%%%%%%%%%%%%%%%%%%%%%%%%

%-- A1 beamer slide ----------------------------------------------------------
\documentclass[final]{beamer}
\usepackage[orientation=portrait,
            scale=1.25,           % font scale factor
            size=a1               % poster format
            ]{beamerposter}

\usepackage{url}
\usepackage{tikz}
\usepackage{hyperref}
\input{macros}
\input{macros-ph}
\input{macros-abstr}
\input{tikzstyles2}
\newcommand{\acm}[3]{#1\rightarrow#2\Rsh#3}
\newcommand{\ac}[3]{$#1\rightarrow#2\Rsh#3$}
\usetheme[facepic=idSmall.jpg]{JDOC}% file path to your id picture

%% Change general color and background
%% http://tango.freedesktop.org/static/cvs/tango-art-tools/palettes/Tango-Palette.svg
 \colorlet{colorprimary}{skyblue1}
 \colorlet{colorsecondary}{skyblue2}
 \colorlet{colortertiary}{skyblue3}
 \colorlet{colorquaternary}{colortertiary!50!black}
 \beamertemplatesolidbackgroundcolor{colorprimary!2}

\usepackage[utf8]{inputenc}

%% Source code environment
\lstset{%
    language=Prolog,
    breaklines=true}

%-- Title and authors of the poster ------------------------------------------
\title{% Poster title
  \Huge{\textbf{A Hybrid Analysis for Reachability Problem}}
}

\author{% Author
  Xinwei Chai, Supervisors: Morgan Magnin, Olivier Roux; \texorpdfstring{\\ Email : \{xinwei.chai, morgan.magnin, olivier.roux\}@ls2n.fr}{}}

\institute{%
  Sp\'{e}cialit\'{e} : Informatique\\ % Research category
  Laboratoire : LS2N\\         % Laboratory name
  \'{E}quipe : MéForBio                 % Team name
}

\begin{document}

%% Source code declaration
\newsavebox{\myLst}
\begin{lrbox}{\myLst}
\begin{lstlisting}[frame=L]
% Rule 1, a node appears always earlier than its predecessor
prior(N1,N2):-parent(N2,N1).
%Rule 2, transitivity
prior(N1,N3):-prior(N1,N2),prior(N2,N3).
%Rule 3, core rule
prior(N1,N2):-node(P1,S1,N1),node(P2,S2,N2),node(P2,S3,N3),
              parent(N1,N3), init(P2,S3), S2!=S3, P1!=P2. 
\end{lstlisting}
\end{lrbox}

\begin{frame}[t]{}

%-- Poster Content -----------------------------------------------------------
\begin{columns}[t]

%% First column
\begin{column}{.45\linewidth}

\section{Introduction}
With increasing quantities of available data provided by new technologies, \textit{e.g.} DNA microarray \cite{marx2013}, Many biological systems are modeled, but a great quantity of them are to be verified whether they are consistent with \textit{a priori} knowledge. 
There is a growing need for expressive modelings and their related high-performance analytic tools. Among various dynamical properties, we focus on reachability. From a biological view, the reachability of one state could represent activation/inhibition of certain gene or synthesis of a protein, while initial state could represent initial observation in an experiment.

In huge biological systems, time/memory-out problem usually falls on traditional model checkers, e.g. Mole \cite{chatain2014characterization} and NuSMV \cite{cimatti2000nusmv}. 
Also, when one demands high precision of prediction, pure static approaches fail because of the partial search in the state space.

\begin{figure}
    \begin{tikzpicture}[line,>=stealth]
        \node [color=gray] (1) {Real system};
        \node [color=blue,right = of 1] (2) {Temporal properties};
        \node [below = of 1] (3) {Partial observation};
        \node [right = of 3] (4) {Model inference};
        \node [right = of 4] (8) {Model};
        \draw [->] (4) -- (8);
        \node [color=blue,right = of 8] (5) {Reachability};
        \node [color = red, below left = 2cm and -1.5cm of 5] (7) {Model Checking};
        \node [color=blue,above = of 2] (6) {Biological \textit{a priori} knowledge};
        %\draw [dashed,->] (1) -- (2);
        \draw [color=blue,->] (2) -- (5);
        \draw [dashed,->] (1) -- (3);
        \draw [->] (3) -- (4);
        %\draw [->] (8) -- (5);
        \draw [color=blue, ->] (6) -- (2);
        \node [below = of 5] (9) {$+$};
        \draw [->] (8) --(9);
        \draw [->] (5) --(9);
        \draw [->] (9) --(7);
        \draw[thick, color=red] (7)--(8);
        \draw[thick,dashed, color=red] (7)--(4);
    \end{tikzpicture}
    \caption{Schema of the methodology, blue texts are the finished processes, red texts shows the work in progress}
\end{figure}

\section{Contribution}
Facing the dilemma given by computation precision demand, we propose a new modeling framework carrying the biological information: Asynchronous Binary Automata Network (ABAN), together with a new model checking approach ASPReach for reachability problems. 
This approach integrates a static analysis Local Causality Graph (LCG) \cite{pauleve2012} related to ABAN and Answer Set Programming (ASP), one sort of paradigm of declarative programming.
The interest of this technique is to have a better conclusiveness than former static approaches and better time-space performance than traditional model-checkers.

\begin{figure}
    \centering
    \begin{tikzpicture}[apdotsimple/.style={apdot},scale=2, every node/.style={scale=2}]
{\tiny
\TSort{(0,0)}{a}{2}{l}
\TSort{(2,0)}{b}{2}{l}
\TSort{(4.5,0)}{c}{2}{l}
}
% with delays
\path[local transitions]

	(c_0) edge node[auto] {\{$a_1, b_1$\}} (c_1)
	(a_0) edge node[auto] {\{$b_0$\}} (a_1)
	(b_0) edge node[auto] {\{$a_0$\}} (b_1)
;
\TState{a_0, b_0, c_0}
\end{tikzpicture}
    \caption{Visualization of ABAN consisting of 3 automata, the gray circles shows the initial state, while the arrows shows the transition, with the conditions of the transition in the braces, e.g. $b_1$ means automaton $b$ is in state 1}
    \label{fig:my_label}
\end{figure}

\begin{figure}
    \centering
    \begin{tikzpicture}[aS,scale=1, every node/.style={scale=2}]   
\scriptsize{  	
  	\startl{c_1};
  	\specl{above}{c_1}{a_1};
  	\link{a_1}{b_0};
  	\edl{b_0};
  	\specl{below}{c_1}{b_1};
	\link{b_1}{a_0};
  	\edl{a_0};
  }
\end{tikzpicture}
    \caption{Visualization of the LCG for analyzing the reachability of $a_1$, with the squares representing local states and small circles representing solution nodes}
\end{figure}
Sole analysis done by LCG leads to inconclusive results, as it checks only the causality of the target state, but is not sure whether there exists a pathway from the initial state to the target state.

We use ASP to do a thorough search without traversing the state space. 
ASP is a prolog-like declarative programming paradigm.
It uses description and constraints of the problem instead of the imperative orders.
\end{column}

%% Second column
\begin{column}{.50\linewidth}


\begin{block}{ASP code to find a good transition order}

\end{block}

\usebox{\myLst} % Source code

For the LCG just shown, Rule 1 gives $b_1\rhd a_1$, $c_1\rhd a_1$, $c_0\rhd b_1$, $b_0\rhd c_1$; Rule 2 gives $c_0\rhd a_1$ and $b_0\rhd a_1$; Rule 3 gives $a_1\rhd b_1$ and $b_1\rhd a_1$ which is impossible, such that there does not exist a trajectory to reach $a_1$ from initial state.

\section{Result and discussions}
We take models of $\lambda$-phage \cite{thieffry1995dynamical} (4 components and 12 transitions), T-cell Receptor (TCR) \cite{saez2007logical} (95 components and 206 transitions) and epidermal growth factor receptor (EGFR) (104 components and 389 transitions) \cite{samaga2009logic} as examples. In every model, we take several automata as input, varying exhaustively their initial states combinations ($2^{|initialStates|}$), take the reachability of the states of some automata as output. We first test the performance of traditional model checkers, in which Mole turns out to be timeout for 6 in 12 outputs, and all timeout for NuSMV in model EGFR. When it comes to Pint, a static model checker, it fails on some instances (Inconclusive) but our approach \textbf{always} find the pathway towards the target state.
\begin{table}
\small{
\begin{tabular}{|c|c|c|c|c|c|c|}
    \hline
  	Model	&\multicolumn{2}{c|}{$\lambda$-phage}	&	  \multicolumn{2}{c|}{TCR} & \multicolumn{2}{c|}{EGFR}  \\
    \hline
    Inputs&\multicolumn{2}{c|}{4}	&	  \multicolumn{2}{c|}{3} & \multicolumn{2}{c|}{13}\\
    \hline
    Outputs&\multicolumn{2}{c|}{4} &	  \multicolumn{2}{c|}{5} & \multicolumn{2}{c|}{12} \\
    \hline
    Total tests&\multicolumn{2}{c|}{$2^4\times 4=64$} & \multicolumn{2}{c|}{$2^3\times 5=40$} & \multicolumn{2}{c|}{$2^{13}\times 12=98,304$}\\
    \hline
    Analyzer  &  Pint       &\textbf{ASPReach}    &  Pint       &\textbf{ASPReach}   &  Pint       &\textbf{ASPReach}             \\
    \hline
    Reachable    & 36(56\%)& 38(59\%)   &  \multicolumn{2}{c|}{16(40\%)}  & 64,282(65.4\%)&74,268(75.5\%)\\
    \hline
    \textbf{Inconclusive} & \textcolor{red}{\textbf{2(3\%)}}&\textcolor{blue}{\textbf{0(0\%)}}& \multicolumn{2}{c|}{0(0\%)}    &\textcolor{red}{\textbf{9,986(10.1\%)}}&\textcolor{blue}{\textbf{0(0\%)}}  \\
    \hline
    Unreachable     &  \multicolumn{2}{c|}{26(41\%)} &  \multicolumn{2}{c|}{24(60\%)} &24,036(24.5\%)&24,036(24.5\%)\\
    \hline
    Total time &  $<1$s       &  $<1$s &  7s       &  40s        & \textbf{9h50min}              & \textbf{3h46min}      \\
    \hline
  \end{tabular}
  \caption{Benchmark on different models, timeout for traditional models and thus not listed}
  }
\end{table}

\section{Ongoing and Future Work}
For now, we are collaborating with a tool for model inference LFIT using machine learning technique \cite{inoue2014learning}, verifying the results of LFIT whether they are consistent with \textit{a priori} knowledge.
In the future research, we are going to work on the modification of models according to \textit{a priori} knowledge and 
also develop analyzer of other dynamical properties.

% References
\subsection{References}
\bibliographystyle{plain}
\bibliography{bib}

\end{column}
\end{columns}

%% End of columns

\end{frame}
\end{document}

